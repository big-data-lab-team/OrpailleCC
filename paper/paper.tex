\documentclass{article}
\usepackage[margin=0.75cm]{geometry}
\usepackage{hyperref}
\usepackage{graphicx}
\usepackage{float}
\usepackage{caption}
\usepackage[lofdepth,lotdepth]{subfig}

\title{Library for Embedded Datastream Algorithm}
\author{Martin, Tristan, Bô}
\begin{document}
\maketitle
\section{Summary}
%Why c++?
%interesting -> more accurate
%Intro plus détaillée
%Détailler FreeRTOS
%œ
Given the increasing number of streaming devices, it has become interesting to
offload stream mining techniques onto these devices. Formerly, connected
devices forwarded their data to a data center where they would be mined
offline.  However, Mining the stream before it leaves the device could help
improving several aspects of the Internet of Thing. Indeed, energy efficiency
can be improved by only forwarding the relevant data and thus saving
communication time. The privacy can be ensured by guaranteeing that the data that
identify a user would never leave the device. Only the mined data would be sent
to the cloud.

OrpailleCC provides a consistent collection of data stream algorithms developed
to be deployed on embedded devices.  All these algorithms rely as little as
possible on operating system related functions such as malloc. When algorithms
could not be implemented otherwise, the library would introduce template
parameters to get the required functions from the user.  All algorithms
initially targeted FreeRTOS \cite{freertos}, a real-time operating system used
in embedded systems, but they should work on any micro-controller with a C++11
compiler. The programming language was chosen for its performance as well as
its popularity in the field.

OrpailleCC addresses classes of problems discussed in \cite{kejariwal2015}. Two
common classes are the \textit{Sampling} and the \textit{Filtering}.  The
\textit{Sampling} covers algorithms that build a representative sample out of a
data stream. Amongst these algorithms, OrpailleCC implements the reservoir
sampling \cite{reservoir_sampling} and one variant, the chained reservoir
sampling \cite{chained_reservoir_sampling}.  The \textit{Filtering} class
groups algorithms that test membership of elements in the data stream in order
to only keep a sub-part of it. The Bloom Filter \cite{bloom} and the Cuckoo
Filter \cite{cuckoo_filter} are two well-tested algorithms that address this
problem.

In addition to the \textit{Sampling} and the \textit{Filtering}, OrpailleCC
provides algorithms for two other classes, \textit{Classification}, and
\textit{Compression}. The \textit{Micro-Cluster Nearest Neighbour} \cite{mc-nn}
is based on the \textit{k-nearest neighbor} to classify a data stream while
catching concept drift.  On the other hand, the Lightweight Temporal
Compression \cite{ltc} and a multi-dimensional variant \cite{ltcd} are two
methods to compress data streams.

The goal of OrpailleCC is to keep including new reliable algorithms to widely
cover as many classes of problems as possible.
\bibliographystyle{plain}
\bibliography{paper}
\end{document}
